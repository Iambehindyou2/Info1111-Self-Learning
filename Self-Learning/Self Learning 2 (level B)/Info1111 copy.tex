\documentclass[a4paper, 11pt]{report}
\usepackage{blindtext}
\usepackage[T1]{fontenc}
\usepackage[utf8]{inputenc}
\usepackage{titlesec}
\usepackage{fancyhdr}
\usepackage{geometry}
\usepackage{fix-cm}
\usepackage[hidelinks]{hyperref}
\usepackage{graphicx}
\usepackage{titlesec}
\usepackage{hyperref}
\usepackage{url}
\usepackage[utf8]{inputenc}
\usepackage{amsmath}
\usepackage{amsfonts}
\usepackage{amssymb}
\usepackage{graphicx}
\usepackage{listings}

\usepackage[english]{babel}

\geometry{ margin=30mm }
\counterwithin{subsection}{section}
\renewcommand\thesection{\arabic{section}.}
\renewcommand\thesubsection{\thesection\arabic{subsection}.}
\usepackage{tocloft}
\renewcommand{\cftchapleader}{\cftdotfill{\cftdotsep}}
\renewcommand{\cftsecleader}{\cftdotfill{\cftdotsep}}
\setlength{\cftsecindent}{2.2em}
\setlength{\cftsubsecindent}{4.2em}
\setlength{\cftsecnumwidth}{2em}
\setlength{\cftsubsecnumwidth}{2.5em}

\titlespacing\section{0pt}{12pt plus 4pt minus 2pt}{0pt plus 2pt minus 2pt}
\titlespacing\subsection{0pt}{12pt plus 4pt minus 2pt}{0pt plus 2pt minus 2pt}
\begin{document}
\titleformat{\section}
{\normalfont\fontsize{15}{0}\bfseries}{\thesection}{1em}{}
\titlespacing{\section}{0cm}{0.5cm}{0.15cm}
\titleformat{\subsection}
{\normalfont\fontsize{13}{0}\bfseries}{\thesubsection}{0.5em}{}
\titlespacing{\section}{0cm}{0.5cm}{0.15cm}

%=============================================================================

\pagenumbering{Alph}
\begin{titlepage}
\begin{flushright}
\includegraphics[width=4cm]{USyd}\\[2cm]
\end{flushright}
\center 
\textbf{\huge INFO1111: Computing 1A Professionalism}\\[0.75cm]
\textbf{\huge 2023 Semester 1}\\[2cm]
\textbf{\huge Self-Learning Report}\\[3cm]

\textbf{\huge Submission number: 1}\\[0.75cm]
 \textbf{Github link: https://github.com/Iambehindyou2/Info1111-Self-Learning}\\[2cm]

{\large
\begin{tabular}{|p{0.35\textwidth}|p{0.55\textwidth}|}
	\hline
	{\bf Student name} & Kailin Zhang\\
	{\bf Student ID} & 5305576006\\
	{\bf Topic} & PHP \\
	{\bf Levels already achieved} & A\\
	{\bf Levels in this report} & B\\
	\hline
\end{tabular}
}
\thispagestyle{empty}
\end{titlepage}
\pagenumbering{arabic}
%=============================================================================
\newpage
\section{Level A: Initial Understanding}
\vspace{5mm}
\subsection{Level A Demonstration}
\begin{itemize}
  \item The Installation of PHP and the IDE/Code editor in order to write in PHP. This includes getting familiar with PHP basic syntax and able to write simple PHP script
  \item Using PHP to connect to simple databases, this may include filtering, displaying and counting data etc
  \item Learning Object-oriented programming of PHP
\end{itemize}

\subsection{Learning Approach}

In the very beginning, before I selected my self-learning topic, I conducted some simple research on all the options available. I discovered that I had an interest in cybersecurity and PHP was a popular language for web development and back-end, so I chose it as my topic. Although I had zero knowledge of PHP, I first watched some brief introduction videos on YouTube to gain some basic knowledge. After understanding what PHP is, I searched for a tutorial on installation. Personally, I believe that learning without practical application is worthless because people will soon forget what they've learned. I like to learn by practising on my own to gain a better understanding. After I installed PHP and the IDE and was ready to go, I found a website called "w3school" that provided detailed knowledge and lectures on PHP. I went through the lectures while referring to a five-hour video on YouTube that talked about PHP. I found that incorporating videos and lectures helped me learn better.

\subsection{Challenges and Difficulties}

I think in general all computer languages, no matter which ones, are a challenge for me because I don’t have experience with using them. After selecting the PHP, I found out that the PHP coding language is like a big ZIP file, it also contains HTML, MYSQL and other languages or forms of coding. For instance, using the level A self-learning as an example, in order for me to create an actual web I have to use HTML. However, at the same time I have to embed the PHP into the file in order for users to input value and also for “me” to receive the value. Connecting the database is like on a different level to be honest, personally speaking I think that is for level B and above. Because it requires knowledge toward using MYSQL and that is like another different set of coding. I already downloaded “DataGrip” and am exploring how to use DataGrip to manipulate SQL files and connect through with PHP.

\subsection{Learning Sources}
\begin{itemize}
  \item \url{https://www.youtube.com/watch?v=OK_JCtrrv-c&t=2640s}{This is a 5 hour video explaining and tutoring on PHP, it helped me a lot with all the syntax and functions of PHP}\cite{PHP}
  \item \url{https://www.youtube.com/watch?v=a7_WFUlFS94}{This is a short video briefly introducing PHP overall}\cite{PHP100}
  \item \url{https://www.bilibili.com/video/BV18x411H7qD/?spm_id_from=333.337.search-card.all.click}{This is also a very detailed tutoring video on PHP, it helped me a lot on making the local host as the server}\cite{HM}
  \item \url{https://www.w3schools.com/php/php_math.asp}{This is a lecture in text, going through all types of math, functions, data type and all other syntax}\cite{W3}
\end{itemize}



%=============================================================================

\newpage

\includegraphics[width=1\textwidth]{1}
\includegraphics[width=1\textwidth]{2}

%=============================================================================
\newpage
\section{Level B: Basic Application}
\subsection{Level B Demonstration}
So my topic that I chose was PHP, so for level b I developed some what knowledge in XAMPP which is a web sever that supports PHP language and also MySQL database. I used XAMPP in level b to generate a local sever for me to use php html css with the MySQL database to create a sign up web page that can store the users' input(usernames and password)
Meanwhile, I have also used Phpadmin to store the login data using MySQL.
\subsection{Application artifacts}

I have created a simple website with a sign-up page that can store user input in a database. The process I followed is as follows:

\begin{enumerate}
    \item Downloaded XAMPP, a cross-platform server, to create a local host on my laptop.
    \item Launched Apache and MySQL inside XAMPP to create a local server that supports PHP and MySQL databases.
    \item Downloaded phpMyAdmin and inserted it into the XAMPP "htdocs" folder to help manage the MySQL database.
    \item Created a MySQL database and user with all privileges using phpMyAdmin.
    \item Created a table under the database structure to store users' input usernames and passwords.
    \item Used Visual Studio Code to create PHP files for connecting to the database and processing the sign-up form.
    \item Incorporated an HTML sign-up form template from the getbootstrap website, and used the "post" method for secure data transmission.
    \item Utilized prepared statements with \texttt{mysqli\_prepare()} to prevent SQL injection attacks.
    \item Implemented error handling for cases when user information is already stored in the database.
\end{enumerate}

The resulting website has a simple structure with a sign-up page that stores user input in the database. If the user's information is already stored in the database, it will detect and output an error; otherwise, it will store the information successfully.

Here is a code snippet from the PHP file for connecting to the MySQL database:

\begin{lstlisting}[language=php]
<?php
// Connection details
$hostname = "SQL;
$username = "k";
$password = "123";
$database = "registration";

// Create connection
$connection = mysqli_connect($hostname, $username, $password, $database);

// Check connection
if (!$connection) {
    die("Connection failed: " . mysqli_connect_error());
}
?>
\end{lstlisting}

\includegraphics[width=0.5\textwidth]{Local}
\includegraphics[width=0.5\textwidth]{PHPadmin}
\includegraphics[width=0.5\textwidth]{getbootstrap}
\includegraphics[width=0.5\textwidth]{XAMPP}

%=============================================================================
\newpage
\bibliographystyle{IEEEtran}
\bibliography{123}


\end{document}


\end{document}
\end{report}
